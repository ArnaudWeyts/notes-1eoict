\documentclass[11pt, a4paper]{report}
\usepackage{tikz, tikz-timing, verbatim}
\usepackage{circuitikz}
\usetikzlibrary{circuits.logic.IEC, positioning}
\usepackage[dutch, english]{babel}
\usepackage[linkcolor=black,urlcolor=blue,citecolor=black]{hyperref}
\usepackage[font=small,format=plain,labelfont=bf,up,textfont=it,up]{caption}
\usepackage[usenames,dvipsnames]{pstricks}
\usepackage[parfill]{parskip}
\hypersetup{colorlinks=true}
\usepackage{epsfig, amsmath, epic, eepic, float, subfig, amsfonts, color, amsthm, textcomp, microtype, graphicx}
\newcommand{\HRule}{\rule{\linewidth}{0.5mm}}

\begin{document}

\begin{center}
\large Voorbereiding Labo 4 digitale technieken

\normalsize Haroen Viaene |
\normalsize 1EO-ICT 10 | 
\normalsize 17 maart 2015
\end{center}

\vspace{3\baselineskip}

\section*{4.3.1. Worst-case-waarde van de weerstand}

\begin{tabular}{| l | l | l |}
  \hline
   & Actief Hoog & Actief laag \\
  \hline
  74LS & $R_{max} = 2.22 k\Omega \rightarrow 2.7 k\Omega$ & $R_{max} = 10 k\Omega$ \\
  74HCT & $R_{max} = 10 k\Omega$ & $R_{max} = 10 k\Omega$ \\
  74HC & $R_{max} = 10 k\Omega$ & $R_{max} = 10 k\Omega$ \\
  40XXBC & $R_{max} = 10 k\Omega$ & $R_{max} = 10 k\Omega$ \\
  \hline
\end{tabular}

\section*{4.3.2. $X = (A + B\cdot C)\cdot \overline{D} + E$}

\subsection*{IEC Poortschema}

\begin{circuitikz}
  \draw
    (2,0) node[european or port] (myor){}
    (myor.in 1) node[anchor=east]{A}


    (0,-1) node[european and port] (myand){}
    (myand.in 1) node[anchor=east]{B}
    (myand.in 2) node[anchor=east]{C}
    (myand.out) -- (myor.in 2)

    (0, -2.3) node[european not port] (mynot){}
    (mynot.in) node[anchor=east]{D}

    (4,-2) node[european and port] (otherand){}
    (otherand.in 1) -- (myor.out)
    (otherand.in 2) -- (mynot.out)

    (6,-3) node[european or port] (otheror){}
    (otheror.in 1) -- (otherand.out)
    (otheror.in 2) node[anchor=east]{E}
    (otheror.out) node[anchor=west]{X}
    ;
\end{circuitikz}

\subsection*{Waarheidstabel}

\begin{tabular}{| c c c c c | c |}
\hline
E & D & C & B & A & X \\
\hline
0 & 0 & 0 & 0 & 0 & 0 \\
0 & 0 & 0 & 0 & 1 & 1 \\
0 & 0 & 0 & 1 & 0 & 0 \\
0 & 0 & 0 & 1 & 1 & 1 \\
\hline
0 & 0 & 1 & 0 & 0 & 0 \\
0 & 0 & 1 & 0 & 1 & 0 \\
0 & 0 & 1 & 1 & 0 & 0 \\
0 & 0 & 1 & 1 & 1 & 0 \\
\hline
0 & 1 & 0 & 0 & 0 & 0 \\
0 & 1 & 0 & 0 & 1 & 1 \\
0 & 1 & 0 & 1 & 0 & 1 \\
0 & 1 & 0 & 1 & 1 & 1 \\
\hline
0 & 1 & 1 & 0 & 0 & 0 \\
0 & 1 & 1 & 0 & 1 & 0 \\
0 & 1 & 1 & 1 & 0 & 0 \\
0 & 1 & 1 & 1 & 1 & 0 \\
\hline
1 & 0 & 0 & 0 & 0 & 0 \\
1 & 0 & 0 & 0 & 1 & 1 \\
1 & 0 & 0 & 1 & 0 & 0 \\
1 & 0 & 0 & 1 & 1 & 1 \\
\hline
1 & 0 & 1 & 0 & 0 & 0 \\
1 & 0 & 1 & 0 & 1 & 1 \\
1 & 0 & 1 & 1 & 0 & 0 \\
1 & 0 & 1 & 1 & 1 & 1 \\
\hline
1 & 1 & 0 & 0 & 0 & 0 \\
1 & 1 & 0 & 0 & 1 & 1 \\
1 & 1 & 0 & 1 & 0 & 1 \\
1 & 1 & 0 & 1 & 1 & 1 \\
\hline
1 & 1 & 1 & 0 & 0 & 0 \\
1 & 1 & 1 & 0 & 1 & 1 \\
1 & 1 & 1 & 1 & 0 & 1 \\
1 & 1 & 1 & 1 & 1 & 1 \\
\hline
\end{tabular}

\subsection*{Logische vergelijking met $NEN$}

\begin{equation*}
  X = (A + B\cdot C)\cdot \overline{D} + E
\end{equation*}
\begin{equation*}
  X = A \cdot \overline{D} + B \cdot C \cdot \overline{D} + E
\end{equation*}
\begin{equation*}
  X = \overline{\overline{A \cdot \overline{D}}\cdot \overline{B \cdot C \cdot \overline{D}}}+ E
\end{equation*}
\begin{equation*}
  X = \overline{\overline{\overline{\overline{A \cdot \overline{D}}\cdot \overline{B \cdot C \cdot \overline{D}}}} \cdot \overline{E}}
\end{equation*}
\begin{equation*}
  X = \overline{\overline{A \cdot \overline{D}}\cdot \overline{B \cdot C \cdot \overline{D}} \cdot \overline{E}}
\end{equation*}

\subsection*{Logische vergelijking met $NOF$}
\begin{equation*}
  X = (A + B\cdot C)\cdot \overline{D} + E
\end{equation*}
\begin{equation*}
  X = A \cdot \overline{D} + B \cdot C \cdot \overline{D} + E
\end{equation*}
\begin{equation*}
  X = \overline{ \overline{A} + D } + \overline{ \overline{B} + \overline{C} + D} + E
\end{equation*}

\subsection*{IEC-poortschema met $NEN$}

\vspace{10 cm}

\subsection*{IEC-poortschema met $NOF$}

\vspace{10 cm}

\end{document}